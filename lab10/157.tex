\documentclass[12pt, a4paper]{article}
\usepackage[T2A]{fontenc}
\usepackage[utf8]{inputenc}
\usepackage[english,russian]{babel}
\usepackage{indentfirst}
\usepackage{amsmath}
\begin{document}

ных чисел, а вторая - только на множестве всех неотрица-

тельных.

\quadИногда функция задается с помощью нескольких фор-

мул, например

$$ f(x) = 
\begin{cases}
    2 ^x \qquad \text{для}x > 0,\hfill \\
    0  \qquad  \text{для}x = 0,\hfill \qquad\qquad\qquad\qquad\qquad(5.1)\\
    x -1\text{для}x < 0. \hfill\\
\end{cases} $$

\quadФункция может быть задана также просто с помощью опи-

сания соответствия. Поставим в соответствие каждому числу

x > 0 число 1, числу 0 - число 0, а каждому х < 0 - число -1.

В результате получим функцию, определенную на всей число-

вой оси и принимающую три значения: 1, 0 и -1. Эта функ-

ция имеет специальное обозначение\textsuperscript{1} sign х и может быть за-

писана с помощью нескольких формул:

$$ sign\quad x = 
\begin{cases}
    1\qquad  \text{для}x > 0,\hfill \\
    0\qquad  \text{для}x = 0,\hfill \qquad\qquad\qquad\qquad\qquad(5.2)\\
    -1\quad  \text{для}x < 0. \\
\end{cases} $$


\quadДругой пример: каждому рациональному числу поставим 

в соответствие число 1, а каждому иррациональному - число 

нуль. Полученная функция называется \emph{функцией {Дирихле}\textsuperscript{2}}

\quadОтметим, что всякая формула является символической 

записью некоторого описанного ранее соответствия, так что, 

в конце концов, нет принципиального различия между зада-

нием функции с помощью формулы или с помощью описа-

ния соответствия; это различие чисто внешнее.

\quadСледует также иметь в виду, что всякая вновь определен-

ная функция, если для нее ввести специальное обозначение,

может служить для определения других функций с по-

мощью формул, включающих этот новый символ.

\quadЕсли речь идет о действительных функциях одного дей-

ствительного аргумента, то для наглядного представления о

характере функциональной зависимости часто строятся гра-

фики функций на координатной плоскости (координатной

называется плоскость, на которой задана прямоугольная де-

картова система координат)

\rule{2cm}{0.4pt}

\textsuperscript{1} Signum(лат.) означает <<знак>>

\textsuperscript{2} Л.Дирихле (1805 -- 1859) -- немецкий математик и ученый

\end{document}
